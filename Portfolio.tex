\documentclass{article}
\usepackage{graphicx}

\begin{document}

\title{Operations and Security portfolio}
\author{James McKenzie}

\maketitle
This a portfolio of work that I have completed and intend to complete in Semester 1 of 2020, as part of my work in the 
Operations and Security team in Project 2
\medskip
This semester, my primary goal is to complete some AWS certifications or some Azure certifications. Im mostly looking into the 
Operations certifications, although some of the software engineering and security ones also look pretty good. My other goal
is to get a better understanding of CI/CD and learn some CI/CD tools such as Spinnaker and Jenkins. This also gives me an opportunity
to learn more of the Rust programming language, as that is one of the rising stars in programming; Being used by both Systems administrators,
and Software developers both. I can use test programs to test my CI/CD testing


\section{Week 01}
The first week of Class is mostly uneventful, but i will document the first week,  just for the sake of completeness. 


\subsection{Class 1}

Today we mainly just set up our machines for the semester. A lot of people seemed somewhat interested in OPs, which i guess is good.
Theres talk of maybe visiting a datacenter and getting some AWS / Azure qualifications, which I am all up for

\subsection{Class 2}
Today we got all the new people into the OPs group. Seem like a good bunch. I showed them around the documentation, got some of them added
into the gitlab documentation repo, and gave a rough overview on what we actually do in the ops department, re: keeping gitlab open, and helping
other projects with any servers or CI/CD that they may need. Rob also came to bother us regarding some issues with both gitlab having certificate
errors, and the server upstairs having some problems mapping its storage. I took a little look at both of these problems, and decided that 
the git issue is more important as many lecturers use gitlab to teach their classes.


\section{Week 02}
\subsection{Class 1}
Today, as faisal and matt were absent, I held a meeting with everyone just to see what everyone is interested in. We have
a lot of people interested in networking, which is a welcome change from neither me or Matt understanding the first thing about
basic networks. It seems a fair few of them aren't overly comfortable in either Linux or Windows Server operating systems, which
I guess they will learn as they work through the Ops group. I know that I definitely want to do something regarding CI/CD as a main
project this semester, and Faisal said he would give me some software to look at in the coming days. I also ended up renewing all of
the certificates today, which took up a lot of time, as the documentation for the cert renewals is broken. There are chain files, which
I still do not understand, which do not work, but I plan to give them a shot later on this week.  It prevents people logging into gitlab
via the automatic authentication via the idp servers. 
\subsection{Class 2}
Today was spent trying to diagnose some issues regarding gitlab, that I definitely over engineered. We had a message from a student
letting me know that he could see a bunch of other repos. I spent about 4 hours trying to brute force find the cause of this, by checking
file permissions and going through pages of bug reports on google regarding this issue. I then came to my senses, and checked what repos
the student was supposed to be able to access, and then I found out that all students in his class were able to access eachothers repos, 
which seems to be the intended behaviour, so I spent the day just going on a wild goose chase looking for bugs and issues that weren't there

\section{Week 03}
\subsection{Class 1}
Today Faisal talked to me about a CI/CD program called Spinnaker, which is one of the heavy hitters in the CI/CD world. I had a look
through it today, and tried installing it on the Polytech computers. I ran into a large number of issues with the install, however, I did
manage to install a package called Spinnaker on my laptop. It was about this time that I realised i was not doing the install right, as its
supposed to be hosted on a cloud platform like AWS or Azure. I decided that I needeed to do some more research before i continued
\subsection{Class 2}

\section{Week 04}
\subsection{Class 1}
This week faisal wanted us to get the Gitlab CI/CD working, which I gave a good attempt, although, I ran into some issues regarding
installing the runner. I also had to spend a while explaining exit codes to my other team members (explaining that all programs return
an integer to the shell, a 0 if the program ran correctly, or a positive integer if the program exited) I think i will get it working next
session though as i made a breakthrough with the gitlab runner.
\subsection{Class 2}
Today I got the gitlab runner working, turns out most of my issues were due to saving the gitlab runner into a drive that wasnt the 
root drive (in this case C:/) and it didn't have access to the \$PATH for some reason or another. I then ran into the problem that my program
would only ever return a success state, however, I fixed this by realising i am not as good at Python as i had expected. I then got bothered
my Rob to try and fix the IDP certificates, so I did some more research on how Chain Files work, But I will try fix this next time I come in
\section{Week 05}
\subsection{Class 1}
So today i "Fixed" the IDP situation through some very roundabout methods, although, IDP is still refusing to work with the gitlab server, I
think this is still an issue with chain files, as I don't really know what certs I am supposed to chain together due to lack of documentation.
The cert generation scripts on the servers are also binaries and not interpreted source code, for whatever reason, which means I have absolutely
no way of determining how to set up the chain files. Thankfully, we are planning to retire the gitlab service soon, and this won't be a problem
again, but I asked a few people in the OPs group if they could have a look at it at some point when they have time, as maybe between us all, 
one of us may be able to crack this problem.
\subsection{Class 2}
Personal Challenge Day! I managed to brick my laptop which had a large number of private keys that I need to access my personal servers.
So I spent today utilising different methods to get the data off my laptop. A big issue is that I encrypt my data at an operating system
level, which meant I needed to use my decrypting programs to get the data back. I achieved this by using a Live USB of Arch Linux (which
coincidentally, is the Operating System that was on my latop), to set up a environment in ram, to mount my drives, do a decrypt on those
mount points, and then change the root into the original root of my operating system. The issue seemed to be that my boot partition seemed
to have had its sector go bad, So, i threw another HDD into my laptop, and will reinstall Arch Linux at some point in the future. While
this was an odd thing to do in an OPs class, I learnt a fair amount regarding how to use Live USBs to influence a system. I might look
for a way to store keys aswell, so that this doesn't happen again, although that might be a pain, considering the sensitive nature of private
keys.
\section{Week 06}
\subsection{Class 1}
There was a big talk about potentially working from home soon, as well as a ticket that came in from the Media Analytics team, who
wanted us to look for a solution to a visitor counter program for their project for when it goes live sometime this year. I had a little
look around, but it seemed that most sites used a backend database to handle that, rather than an independant software solution. They
said they would appreciate it if i keeped looking into it.
\subsection{Class 2}
With this Covid 19 Stuff getting somewhat serious, I spent all of today making sure I have the required Keys, and files so I can continue 
working from home. I also fiddled around a little bit trying to get my AWS docker linking to work, although, I didn't have much luck, as 
instead of it giving me a command, It gave me a key? I should probably sort this out after the break
\section{Week 07}
\subsection{Class 1}
I spent today at home, working through some AWS setup. It is still an absolute hell to setup, with paths and conflicting docker instructions, 
as well as multiple verions of the AWS terminal. I think I will work more on this later, as it is giving me a hell of a headache. Unfortunately
i cannot progress to doing more stuff with Spinnaker, as i need an AWS setup to do so.
\subsection{MidSemBreak}
Over the Midsemester break, I worked on a website, https://staging.james-mckenzie.tech/ For this, I used an online CI/CD solution,
called zeit, using a github bot called now. I mainly am doing this to get a better understanding of development workflow, and also so I
have a portfolio of work for when I go out into the workforce and start looking for work.
\subsection{Class 2}
We are finally back from break! While its been fun, Im glad to get back into the swing of things. Today I had a chat to
Amazon in regard to my account being locked for some reason. It turns out I had locked myself out of my account due to a number
of unfortunate typos. Lets hope I can make some progress faster in the coming weeks, as i really need to complete my Spinnaker setup,
as well as getting my AWS OPs certification. 

\section{Week 08}
Unfortunately, Due to working at the hospital, and a handful of classwork, I was unable to complete any OPs work this week. 





\end{document}